\chapter{Introduction}
\label{cha:introduction}

\begin{quote}
   This chapter introduces the PyLinear Python extension and outlines the rest
   of the document.
\end{quote}

PyLinear is a set of extensions to the Python programming language
which allows Python programmers to efficiently manipulate matrices and
vectors, the primary objects of linear algebra.  It allows real and
complex arithmetic, currently only in double precision.  Dense as well
as two types of sparse matrices are supplied, and a large variety of
numerical algorithms, from eigensolvers, to singular value
decomposition, direct sparse solvers to sparse eigensolvers are also
furnished as part of an ever-growing standard library.

PyLinear's programming interface is similar to that of Numerical
Python and numarray to ease porting, but differs in a few key
aspects. The most notable such aspect is matrix multiplication. While
the term \code{A*B} in Numeric means element-by-element
multiplication, PyLinear changes this to mean conventional
matrix-matrix and matrix-vector multiplication, to match customary
uses in scientific computing, and following the example of languages
such as Matlab\footnote{This has one important gotcha. Vector-vector
multiplication (also known as the dot product) is \emph{not}
associative, i.e. if $a$, $b$ and $c$ are vectors, then typically
$(a\cdot b)\cdot c\not=a\cdot(b\cdot c)$. (Note that the type of the
parenthesized expression is scalar.) PyLinear will not reject code
such as \code{a*b*c}, but its meaning is inherently undefined. Matlab
gets around this limitation by introducing column and row vectors.}.
Chapter \ref{cha:diff-pylinear-numpy} is dedicated to highlighting the
differences between PyLinear and NumPy and its desecendents.

In very simple terms, PyLinear is a mapping of the operators supplied by
Boost.UBlas into Python using the Boost.Python library.  This has two
implications that balance each other. First, PyLinear is no speed demon. It
does have the right asymptotic complexity guarantees (i.e. operations that
ought to be linear-time in fact are). That's the bad news. The good news is
that since PyLinear is essentially a scripting language for Boost.UBlas, it is
sheepishly easy to convert a slow inner loop from Python into C++, without
losing much abstraction: The matrix and vector types as well as most operations
are available in C++ with only slightly more difficult syntax than in Python.
But if that is so, why would you want to use Python in the first place? Because
it's high-level, safe and does not require the sometimes lengthy compile times
of C++.  And you need to convert \emph{only} that slow inner loop!  Since you,
too, can use Boost.Python to bind that inner loop to Python (and still use
PyLinear's facilities), there's no real need to move the whole system into C++.
That way, Python can be the convenient and safe prototyping language for large
computation systems written in C++.
