\chapter{Installation}
\label{cha:installation}

\begin{quote}
   This chapter helps you install PyLinear onto your computer.
\end{quote}

\section{Checking prerequisites}
\label{sec:checking-prerequisites}

The first step in installing PyLinear is to make sure that you have the right
software installed on your computer. You will need the following:

\begin{itemize}
  \item Python is of course the most important prerequisite. 
    Version 2.3 or newer will work.

    You can find Python at \url{http://www.python.org}.
  \item The Boost libraries.
    Versions 1.33 and up work fine. Section \ref{sec:install-bpl}
    will help you with the installation of this prerequisite.

    You can find Boost at \url{http://www.boost.org}.
  \item The Boost Numeric library bindings. 
    Section \ref{sec:install-bindings} will help you with this.

    You can find the Boost Numeric bindings at
    \url{http://news.tiker.net/software/boost-bindings}.
  \item A good enough C++ compiler. GCC version 3.3 and up work fine.

    GCC versions 4.0 and better will compile PyLinear using much less memory
    and generally much faster than the older 3.x series.  Using them is highly
    recommended. Note however that GCC 4.1 has a
    bug that affects Boost.UBlas.
    (see \url{http://gcc.gnu.org/bugzilla/show_bug.cgi?id=28016})

    GCC can be found at \url{http://gcc.gnu.org}.
\end{itemize}

Optionally, you may install the following libraries to augment
PyLinear's functionality:

\begin{itemize}
  \item 
    The \em{Basic Linear Algebra Subprograms}, better known as the BLAS. This will
    not enable any new functionality by itself, but is a prerequisite for many
    of the following libraries.

    You can find the original Fortran BLAS at \url{http://netlib.org/blas}.  A
    good, generic implementation is ATLAS at \url{http://math-atlas.sf.net}.

  \item 
    The \em{Linear Algebra Package}, better known as \em{LAPACK}. This will
    enable a few extra operations on dense matrices, such as finding
    eigenvalues or the singular value decomposition.

    You can find LAPACK at \url{http://netlib.org/lapack}.
    ATLAS now includes a LAPACK implementation.

  \item 
    \em{UMFPACK}, which requires BLAS, provides an efficient direct solver for
    linear systems involving sparse matrices.

    As of late, most major Linux distributions package UMFPACK as part of a
    bigger package called UFsparse. While this is (in my opinion) not such a
    great idea, it's a fact that PyLinear has to live with.

    Newer versions of PyLinear therefore default to using UFsparse,
    but will still run with just UMFPACK (and COLAMD) installed.
    
    Unfortunately, versions of the Boost Numeric Bindings prior to
    release 2006-04-30 (from the PyLinear's author's web site) 
    defaulted to using an include file \file{umfpack/umfpack.h},
    which is not possible within UFsparse, where UMFPACK headers
    are typically installed under \file{/usr/include/ufsparse/umfpack.h}.
    The above-mentioned release fixes this, but also requires you
    to mention the full path to \file{umfpack.h} in \file{siteconf.py}.

    BIG FAT WARNING: If you get the Boost Numeric Bindings from CVS,
    you will have to make this change yourself.

    You can find UMFPACK at \url{http://www.cise.ufl.edu/research/sparse/}.

  \item 
    \em{ARPACK}, which requires both BLAS and LAPACK, allows the solving of
    sparse eigenvalue problems.

    If you are planning on using ARPACK, please also read about the
    this \citetitle[http://news.tiker.net/node/373]{bug} that might
    require you to patch ARPACK in order to use PyLinear with ARPACK.
    (If you don't use it, you may get invalid results or inexplicable
    crashes. Do yourself the favor.)

    You can find ARPACK at \url{http://www.caam.rice.edu/software/ARPACK/}.

  \item 
    \em{DASKR}, a well-known package for solving Differential-Algebraic
    Equations, or DAEs for short. DAEs are a generalization of Ordinary
    Differential Equations, or ODEs for short.

    This functionality is available as part of the toybox (see Chapter
    \ref{cha:numerics}) for what the toybox is), but it's there if you need it.

    For ease of compilation, the DASKR source is packaged with PyLinear. To
    enable its use, simply go to the subdirectory \file{fortran/daskr}, type
    \program{./build.sh}, watch for its successful completion and uncomment the
    default options relating to DASKR in \file{siteconf.py}.

    You can find DASKR at \url{http://netlib.org/ode}.
\end{itemize}

PyLinear allows you to query at runtime which of these packages are available,
see Section \ref{sec:querying-functionality}.

\section{Installing Boost Python}

\label{sec:install-bpl}

Installing the Boost libraries in a way that is suitable for PyLinear
is, unfortunately, a non-straightforward process, at least if you are
doing it for the first time. This section describes that process.

There is a bit of good news, though: If you are lucky enough to be
using the Debian flavor of Linux or one of its derivatives, you may
simply type \code{aptitude install libboost-python-dev} and ignore the
rest of this section.

Otherwise, you must follow these steps:

\begin{itemize}
  \item Download a Boost release.
  \item Download and install Boost.Jam, a build tool.
  \item Build Boost, such as by typing
    \begin{verbatim}
      bjam -sPYTHON_ROOT=/usr -sPYTHON_VERSION=2.4 \
        -sBUILD="release <runtime-link>dynamic <threading>multi"
    \end{verbatim}

    (You may have to adapt \code{PYTHON_ROOT} and
    \code{PYTHON_VERSION} depending on your system.)
  \item Check the directory 
    \begin{verbatim}
      boost/bin/boost/libs/python/build/libboost_python.so...
        .../gcc/release/shared-linkable-true/threading-multi
    \end{verbatim}
    and find \file{libboost_python*.so}. (Don't copy the dots--they are only
    there to make the line fit on this page.) Copy these files to somewhere
    on your dynamic linker path, for example:
    \begin{itemize}
    \item \file{/usr/lib}
    \item a directory on \envvar{LD_LIBRARY_PATH}
    \item or something mentioned in \file{/etc/ld.so.conf}.
    \end{itemize}
    You should also create a symbolic link called \file{libboost_python.so}
    to the main \file{.so} file.

  \item Run \program{ldconfig}.
\end{itemize}

\section{Installing the Boost UBlas Bindings}

\label{sec:install-bindings}
This part is, fortunately, very easy. Just go to \url{http://news.tiker.net/software/boost-bindings},
download the current snapshot and extract it somewhere, for example by typing
\begin{verbatim}
  tar xvfz boost-bindings-NNNNNNNN.tar.gz
\end{verbatim}
Then remember the path where you unpacked it for the next step.

If you get the Boost Numeric Bindings from CVS, please read the section
on installing UMFPACK/UFsparse in Section \ref{sec:checking-prerequisites}.

\section{Installing PyLinear}

As a first step, copy the file \file{siteconf-template.py} to
\file{siteconf.py} and open an editor on that file. You will
see a bunch of variables that you may customize to adapt
PyLinear to your system. First of all, there are a few variables
that are named \var{HAVE_XXX}, such as \var{HAVE_BLAS}. They
all default to \code{False}. If you have the corresponding library
available, set that variable to \code{True}.

For each library that you have answered \code{True} above, you may
need to state in which directories to find the header files (in
\var{XXX_INCLUDE_DIRS}), the libraries (in \var{XXX_LIBRARY_DIRS})
and finally, if the libraries are named in some nonstandard fashion,
you may also have to change the library names to link against (in
\var{XXX_LIBRARIES}). The defaults work at least with Debian Linux. 

These above instructions apply to all prerequisite libraries. Here are
a few hints for specific libraries:
\begin{itemize}
\item For Boost, set \var{BOOST_INCLUDE_DIR} to the directory where
  the root of your boost tree.  Typically, it ends in
  \file{boost}. For \var{BOOST_LIBRARY_DIRS}, give the path where you
  put the \code{libboost_python*.so} files. Finally, you should
  usually leave BPL_LIBRARIES unchanged and make a symbolic link from
  \file{libboost_python.so} to the actual (non-symlink) \file{.so}
  file.
\item For the Boost bindings, just insert the path where you unpacked
  them--No further installation is required.
\item Here's an extra trick for BLAS and LAPACK if you are using
  Debian: If you install lapack and blas, make sure to install the
  versions ending in ``3'' (i.e. \code{blas3-dev} and
  \code{atlas3-dev}), and also install ``\code{atlas3-ARCH}-dev'',
  where \code{ARCH} is your processor architecture. Debian will then
  automatically activate an accelerated BLAS for your computer.
\end{itemize}

Then, type
\begin{verbatim}
  python setup.py build
\end{verbatim}
and wait what happens. This will compile PyLinear, which, depending on
your compiler, will take a little while. Once this step completes, type
\begin{verbatim}
  su -c "python setup.py install"
\end{verbatim}

As a final step, you may change into the \file{test/} subdirectory
and execute
\begin{verbatim}
  python test_matrices.py
\end{verbatim}
This will execute PyLinear's unit test suite. All tests should
run fine, outputting a long line of dots and "OK" as the last line
of output.

Congratulations! You have now successfully installed PyLinear.
