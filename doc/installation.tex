\chapter{Installation}
\label{cha:installation}

\begin{quote}
   This chapter gives you an idea of how to install PyLinear on your 
   computer.
\end{quote}

First of all, let's talk about the software that PyLinear needs to run
or can use to be more useful.
\begin{itemize}
  \item Boost, from \url{http://www.boost.org}. Required.

    Download from CVS, or use version 1.33 (which, as of this writing, 
    is still to be released).
  \item Boost numerics bindings. Required.

    Use the tarball from \url{http://news.tiker.net/software/ublas-bindings}.
  \item BLAS. Optional, but is a prerequisite of all the more
    advanced software. 

    Any machine-specific BLAS should be fine. If your machine doesn't come
    with one, use ATLAS, from \url{http://atlas.sf.net}
  \item LAPACK. Optional, gives you eigenvalues, SVD and such of
    dense matrices.

    Requires BLAS.

    Use the one that came with your computer, the one that came with 
    ATLAS, or the Fortran one from \url{http://www.netlib.org/}.

  \item UMFPACK. Optional, gives you a direct solver for 
    \class{SparseExecuteMatrix} instances.

    Requires BLAS.

    Download from \url{http://www.cise.ufl.edu/research/sparse/umfpack} 
  \item ARPACK. Optional, gives you an eigensolver for
    \class{MatrixOperator} instances.

    Requires BLAS, LAPACK.

    Download from \url{http://www.caam.rice.edu/software/ARPACK/}.
\end{itemize}

Now, copy the file \texttt{siteconf-template.py}, which is distributed
in the PyLinear tarball, to \texttt{siteconf.py} and modify it to suit
your environment. Especially, modify the \texttt{HAVE_SOMETHING} items
to \constant{True} if you have the given piece of software.  Then,
make sure that the \texttt{SOMETHING_LIBRARY_DIRS} and
\texttt{SOMETHING_INCLUDE_DIRS} (if applicable) reflect where the
library binaries (\texttt{libSOMETHING.so} or \texttt{libSOMETHING.a})
and the header files (if applicable) are to be found. If the library
name is different on your machine, also edit
\texttt{SOMETHING_LIBRARIES}.

Lastly, unless you want to wait \em{years} on that compilation, make
sure that the \texttt{OPT} variable in
\texttt{/usr/lib/python2.n/config} does \em{not} contain a
\texttt{-g}.

After all this preparation, type \texttt{python setup.py build}, wait
for the compile to finish successfully, and then \texttt{su -c "python
setup.py install"}.  If all went according to plan, you now have a 
working PyLinear installation.
