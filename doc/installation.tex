\chapter{Installation}
\label{cha:installation}

\begin{quote}
   This chapter helps you install PyLinear onto your computer.
\end{quote}

\section{Checking prerequisites}

The first step in installing TagPy is to make sure that you have the
right software installed on your computer. You will need the
following:

\begin{itemize}
  \item The \citetitle[http://boost.org]{Boost libraries}. 
    Versions 1.33 and up work fine. Section \ref{sec:install-bpl}
    will help you with the installation of this prerequisite.
  \item The
    \citetitle[http://news.tiker.net/software/boost-bindings]{Boost
    UBlas library bindings}. Section \ref{sec:install-bindings} will
    help you with this.
  \item A good enough C++ compiler. \citetitle[http://gcc.gnu.org]{gcc}
    version 3.3 and up work fine.
\end{itemize}

Optionally, you may install the following libraries to augment
PyLinear's functionality:

\begin{itemize}
  \item The \citetitle[http://netlib.org/blas]{Basic Linear
    Algebra Subprograms}, better known as the BLAS. This will
    not enable any new functionality by itself, but is a
    prerequisite for the following libraries.
  \item The \citetitle[http://netlib.org/lapack]{Linear Algebra
    Package}, better known as LAPACK. This will enable a few
    extra operations on dense matrices, such as finding eigenvalues,
    the singular value decomposition or the inverse.
  \item \citetitle[http://www.cise.ufl.edu/research/sparse/umfpack/]{UMFPACK},
    which requires BLAS, will provide an efficient direct solver
    for linear systems involving sparse matrices.
  \item \citetitle[http://www.caam.rice.edu/software/ARPACK/]{ARPACK},
    which requires both BLAS and LAPACK, allows the solving of
    sparse eigenvalue problems.
\end{itemize}

\section{Installing Boost Python}

\label{sec:install-bpl}

Installing the Boost libraries in a way that is suitable for PyLinear
is, unfortunately, a non-straightforward process, at least if you are
doing it for the first time. This section describes that process.

There is a bit of good news, though: If you are lucky enough to be
using the Debian flavor of Linux or one of its derivatives, you may
simply type \code{aptitude install libboost-python-dev} and ignore the
rest of this section.

Otherwise, you must follow these steps:

\begin{itemize}
  \item Download a Boost release.
  \item Download and install Boost.Jam, a build tool.
  \item Build Boost, such as by typing
    \begin{verbatim}
      bjam -sPYTHON_ROOT=/usr -sPYTHON_VERSION=2.4 -sBUILD="release <runtime-link>dynamic <threading>multi"
    \end{verbatim}

    (You may have to adapt \code{PYTHON_ROOT} and
    \code{PYTHON_VERSION} depending on your system.)
  \item Check the directory 
    \begin{verbatim}
      boost/bin/boost/libs/python/build/libboost_python.so/gcc/release/shared-linkable-true/threading-multi
    \end{verbatim}
    and find \file{libboost_python*.so}. Copy these files to somewhere
    on your dynamic linker path, for example:
    \begin{itemize}
    \item \file{/usr/lib}
    \item a directory on \envvar{LD_LIBRARY_PATH}
    \item or something mentioned in \file{/etc/ld.so.conf}.
    \end{itemize}
    You should also create a symbolic link called \file{libboost_python.so}
    to the main \file{.so} file.

  \item Run \program{ldconfig}.
\end{itemize}

\section{Installing the Boost UBlas Bindings}

\label{sec:install-bindings}


\section{Installing PyLinear}

As a first step, copy the file \file{siteconf-template.py} to
\file{siteconf.py} and open an editor on that file. You will
see a bunch of variables that you may customize to adapt
PyLinear to your system. First of all, there are a few variables
that are named \var{HAVE_XXX}, such as \var{HAVE_BLAS}. They
all default to \code{False}. If you have the corresponding library
available, set that variable to \code{True}.

For each library that you have answered \code{True} above, you may
need to state in which directories to find the header files (in
\var{XXX_INCLUDE_DIRS}), the libraries (in \var{XXX_LIBRARY_DIRS})
and finally, if the libraries are named in some nonstandard fashion,
you may also have to change the library names to link against (in
\var{XXX_LIBRARIES}).
