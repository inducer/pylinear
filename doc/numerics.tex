\chapter{Numerics with PyLinear}
\label{cha:numerics}

\begin{quote}
   This chapter introduces the numerical algorithms available
   in PyLinear.
\end{quote}

PyLinear features six different modules of numerical algorithms:
\begin{itemize}
  \item \module{pylinear.operation} uses the previously introduced 
    notion of an \class{Operator} and offers several implementations 
    of the concept. It is also the main module of linear
    algebra computational routines in PyLinear. It offers a
    comprehensive set of linear algebra primitives, such as
    determinants, decompositions, linear solves, eigenvalue finding
    and the like. While the \class{Operator}-based functions have
    been described in Chapter \ref{cha:matrixfree}, the conceptually
    simpler function-call interfaces are described in Section
    \ref{sec:matrix-computations}.
  \item \module{pylinear.linear_algebra} is a compatibility module 
    which aims for complete exchangeability with NumPy's \module{LinearAlgebra}.
    It offers a high-level subset of \module{pylinear.operation} 
    with less exposed detail. See its documentation, which was made
    available as part of NumPy and numarray.
  \item \module{pylinear.mpi} will provide an interface between
   MPI and PyLinear, but is not yet written.
  \item \module{pylinear.toybox} serves as a staging area for
    the above modules and has an unspecified interface that may 
    change at any time. Look in the source to find experimental
    algorithms that may solve your problems, but be warned that
    these may disappear or change at any time.
\end{itemize}

\section{Matrix computations}

\begin{funcdesc}{solve_linear_system}{matrix, rhs}
  Solves the linear system \code{matrix*solution=rhs}. Uses LU
  decomposition or UMFPACK, depending on availability and sparseness
  of \var{matrix}.
\end{funcdesc}
\begin{funcdesc}{solve_linear_system_cg}{matrix, rhs}
  Solves the linear system \code{matrix*solution=rhs}, where \var{matrix} is
  symmetric and positive definite. Uses the Hestenes-Stiefel Conjugate
  Gradient method. See also \class{CGOperator}.
\end{funcdesc}
\begin{funcdesc}{cholesky}{matrix}
  Returns the Cholesky decomposition $L$ of \var{matrix}. If we let $A$
  be equal to \var{matrix}, then $L$ satisfies $A=L L^H$.
\end{funcdesc}
\begin{funcdesc}{lu}{matrix}
  Returns a tuple \code{(l, u, perm, sign)} that represents the LU
  decomposition of \var{matrix}.

  Let $A$ be equal to \var{matrix}, $L$ equal to \var{l}, $U$ equal
  to \var{u} and $P$ equal to a permutation matrix with $P_{i,j}=1$
  iff \code{perm[i]=j}. Then the LU decomposition satisfies $LU=PA$.

  See also \function{make_permutation_matrix}, which turns \var{perm}
  into a matrix like $P$.
\end{funcdesc}
\begin{funcdesc}{eigenvalues}{matrix}
  Returns a sequence (of unspecified type) that contains all eigenvalues
  of \var{matrix}. Requires LAPACK.
\end{funcdesc}
\begin{funcdesc}{diagonalize}{matrix}
  Returns a tuple \code{(vr, w)} of a matrix \var{vr} and a vector \var{w}.

  Let $A$ be equal to \var{matrix}, $V$ equal to \var{vr}, and $D$ equal
  to a \var{u} and $P$ equal to a permutation matrix with $P_{i,j}=1$
  iff \code{perm[i]=j}. Then the LU decomposition satisfies $LU=PA$.
\end{funcdesc}


\section{Convenient helpers}
%\section{Parallel PyLinear}

%Does not exist yet. :)
