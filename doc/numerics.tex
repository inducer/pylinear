\chapter{Numerical algorithms}
\label{cha:numerics}

\begin{quote}
   This chapter introduces the advanced numerical algorithms available
   in PyLinear.
\end{quote}

PyLinear features six different modules of numerical algorithms:
\begin{itemize}
  \item \module{pylinear.operation} introduces the notion of a
    \class{Operator} and offers several implementations of the 
    concept. It is also the main module of linear
    algebra computational routines in PyLinear. It offers a
    comprehensive set of linear algebra primitives, such as
    determinants, decompositions, linear solves, eigenvalue finding
    and the like.
  \item \module{pylinear.linear_algebra} is a compatibility module 
    which aims for complete exchangability with NumPy's \module{LinearAlgebra}.
    It offers a high-level subset of \module{pylinear.operation} 
    with less exposed detail.
  \item \module{pylinear.mpi} will provide an interface between
    MPI and PyLinear, but is not yet written.
  \item \module{pylinear.toybox} serves as a staging area for
    the above modules and has an unspecified interface that may 
    change at any time. Look in the source to find experimental
    algorithms that may solve your problems, but be warned that
    these may disappear or change at any time.
\end{itemize}

\section{The \class{Operator} concept}

An \class{Operator}\index{Operator} in the context 
\section{Matrix computations}
\section{Convenient helpers}
\section{Parallel PyLinear}

Does not exist yet. :)
