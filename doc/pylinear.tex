\documentclass[hyperref]{manual}

% latex2html doesn't know [T1]{fontenc}, so we cannot use that:(

\usepackage{amsmath}
\usepackage[latin1]{inputenc}
\usepackage{textcomp}

% mark internal comments
% for any published version switch to the second (empty) definition of the macro!
% \newcommand{\remark}[1]{(\textbf{Note to authors: #1})}
\newcommand{\remark}[1]{}

\title{PyLinear}

\author{Andreas Kl\"ockner}

\authoraddress{Division of Applied Mathematiccs, Brown University\\Campus Box F, Providence, RI 02912, USA}

\date{March 22, 2005}

% (software) release version;
\release{0.92}
% this is used to define the \version macro
\setshortversion{0.92}

\makeindex

\begin{document}

\maketitle

\begin{abstract}
  \noindent
  For the Python langugage, there is no shortage of packages that
  provide matrices and operations on those. Numerical Python and
  numarray and the new SciPy Core are good examples.

  However, few of these packages focus exclusively on Linear Algebra,
  and even fewer provide crucial tools for numerical analysis, such as
  sparse matrices.

  PyLinear comes in to fix this deficiency, and also, by using the 
  Boost Python and UBlas libraries, aims for easy extensibility on the C++ side.
\end{abstract}

\tableofcontents

\chapter{Introduction}
\label{cha:introduction}

\begin{quote}
   This chapter introduces the pylinear Python extension and outlines the rest
   of the document.
\end{quote}

PyLinear is a set of extensions to the Python programming language
which allows Python programmers to efficiently manipulate matrices and
vectors, the primary objects of linear algebra.  It allows real and
complex arithmetic, currently only in double precision.  Dense as well
as two types of sparse matrices are supplied, and a large variety of
numerical algorithms, from eigensolvers, to singular value
decomposition, direct sparse solvers to sparse eigensolvers are also
furnished as part of an ever-growing standard library.

PyLinear's programming interface is similar to that of Numerical
Python and numarray to ease porting, but differs in a few key
aspects. The most notable such aspect is matrix multiplication. While
the term \code{A*B} in Numeric means element-by-element
multiplication, PyLinear changes this to mean conventional
matrix-matrix and matrix-vector multiplication, to match customary
uses in scientific computing, and following the example of languages
such as Matlab.  Chapter \ref{cha:diff-pylinear-numpy} is dedicated to
highlighting the differences between PyLinear and NumPy and its
desecendents.

In very simple terms, PyLinear is no more than a mapping of the
operators supplied by Boost.UBlas into Python using the Boost.Python
binding library.  This has two implications that balance each
other. First, PyLinear is no speed demon. It does have the right
asymptotic complexity guarantees (i.e. operations that ought to be
linear-time in fact are), but often, due to Python's interpreted
nature, the constants are pretty large. That's the bad news. The good
news is that since PyLinear is essentially a scripting language for
Boost.UBlas, it is sheepishly easy to convert a slow inner loop from
Python into C++, without losing much abstraction: The matrix and
vector types as well as most operations are available in C++ with only
slightly more difficult syntax than in Python. But if that is so, why
would you want to use Python in the first place? Because it's
high-level, safe and does not require the sometimes lengthy compile
times of C++.  And you need to convert \emph{only} that slow inner
loop!  Since you, too, can use Boost.Python to bind that inner loop to
Python (and still use PyLinear's facilities), there's no real need to
move the whole system into C++. That way, Python can be the convenient
and safe prototyping language for large computation systems written in
C++.

\chapter{Installation}
\label{cha:installation}

\begin{quote}
   This chapter helps you install PyLinear onto your computer.
\end{quote}

\section{Checking prerequisites}

The first step in installing TagPy is to make sure that you have the
right software installed on your computer. You will need the
following:

\begin{itemize}
  \item \citetitle[http://www.python.org]{Python} is of course
    the most important prerequisite. Version 2.3 or newer will work.
  \item The \citetitle[http://boost.org]{Boost libraries}. 
    Versions 1.33 and up work fine. Section \ref{sec:install-bpl}
    will help you with the installation of this prerequisite.
  \item The
    \citetitle[http://news.tiker.net/software/boost-bindings]{Boost
    UBlas library bindings}. Section \ref{sec:install-bindings} will
    help you with this.
  \item A good enough C++ compiler. \citetitle[http://gcc.gnu.org]{gcc}
    version 3.3 and up work fine.
\end{itemize}

Optionally, you may install the following libraries to augment
PyLinear's functionality:

\begin{itemize}
  \item The \citetitle[http://netlib.org/blas]{Basic Linear
    Algebra Subprograms}, better known as the BLAS. This will
    not enable any new functionality by itself, but is a
    prerequisite for the following libraries.
  \item The \citetitle[http://netlib.org/lapack]{Linear Algebra
    Package}, better known as LAPACK. This will enable a few
    extra operations on dense matrices, such as finding eigenvalues,
    the singular value decomposition or the inverse.
  \item \citetitle[http://www.cise.ufl.edu/research/sparse/umfpack/]{UMFPACK},
    which requires BLAS, will provide an efficient direct solver
    for linear systems involving sparse matrices.
  \item \citetitle[http://www.caam.rice.edu/software/ARPACK/]{ARPACK},
    which requires both BLAS and LAPACK, allows the solving of
    sparse eigenvalue problems.
\end{itemize}

\section{Installing Boost Python}

\label{sec:install-bpl}

Installing the Boost libraries in a way that is suitable for PyLinear
is, unfortunately, a non-straightforward process, at least if you are
doing it for the first time. This section describes that process.

There is a bit of good news, though: If you are lucky enough to be
using the Debian flavor of Linux or one of its derivatives, you may
simply type \code{aptitude install libboost-python-dev} and ignore the
rest of this section.

Otherwise, you must follow these steps:

\begin{itemize}
  \item Download a Boost release.
  \item Download and install Boost.Jam, a build tool.
  \item Build Boost, such as by typing
    \begin{verbatim}
      bjam -sPYTHON_ROOT=/usr -sPYTHON_VERSION=2.4 -sBUILD="release <runtime-link>dynamic <threading>multi"
    \end{verbatim}

    (You may have to adapt \code{PYTHON_ROOT} and
    \code{PYTHON_VERSION} depending on your system.)
  \item Check the directory 
    \begin{verbatim}
      boost/bin/boost/libs/python/build/libboost_python.so/gcc/release/shared-linkable-true/threading-multi
    \end{verbatim}
    and find \file{libboost_python*.so}. Copy these files to somewhere
    on your dynamic linker path, for example:
    \begin{itemize}
    \item \file{/usr/lib}
    \item a directory on \envvar{LD_LIBRARY_PATH}
    \item or something mentioned in \file{/etc/ld.so.conf}.
    \end{itemize}
    You should also create a symbolic link called \file{libboost_python.so}
    to the main \file{.so} file.

  \item Run \program{ldconfig}.
\end{itemize}

\section{Installing the Boost UBlas Bindings}

\label{sec:install-bindings}
This part is, fortunately, very easy. Just go to \url{http://news.tiker.net/software/boost-bindings},
download the current snapshot and extract it somewhere, for example by typing
\begin{verbatim}
  tar xvfz boost-bindings-NNNNNNNN.tar.gz
\end{verbatim}
Then remember the path where you unpacked it for the next step.

\section{Installing PyLinear}

As a first step, copy the file \file{siteconf-template.py} to
\file{siteconf.py} and open an editor on that file. You will
see a bunch of variables that you may customize to adapt
PyLinear to your system. First of all, there are a few variables
that are named \var{HAVE_XXX}, such as \var{HAVE_BLAS}. They
all default to \code{False}. If you have the corresponding library
available, set that variable to \code{True}.

For each library that you have answered \code{True} above, you may
need to state in which directories to find the header files (in
\var{XXX_INCLUDE_DIRS}), the libraries (in \var{XXX_LIBRARY_DIRS})
and finally, if the libraries are named in some nonstandard fashion,
you may also have to change the library names to link against (in
\var{XXX_LIBRARIES}). The defaults work at least with Debian Linux. 

These above instructions apply to all prerequisite libraries. Here are
a few hints for specific libraries:
\begin{itemize}
\item For Boost, set \var{BOOST_INCLUDE_DIR} to the directory where
  the root of your boost tree.  Typically, it ends in
  \file{boost}. For \var{BOOST_LIBRARY_DIRS}, give the path where you
  put the \code{libboost_python*.so} files. Finally, you should
  usually leave BPL_LIBRARIES unchanged and make a symbolic link from
  \file{libboost_python.so} to the actual (non-symlink) \file{.so}
  file.
\item For the Boost bindings, just insert the path where you unpacked
  them--No further installation is required.
\item Here's an extra trick for BLAS and LAPACK if you are using
  Debian: If you install lapack and blas, make sure to install the
  versions ending in ``3'' (i.e. \code{blas3-dev} and
  \code{atlas3-dev}), and also install ``\code{atlas3-ARCH}-dev'',
  where \code{ARCH} is your processor architecture. Debian will then
  automatically activate an accelerated BLAS for your computer.
\end{itemize}

Then, type
\begin{verbatim}
  python setup.py build
\end{verbatim}
and wait what happens. This will compile PyLinear, which, depending on
your compiler, will take around 400 MB of your computers memory. If you
do not have this much physical memory, this might be quite slow due to
swapping. Once this step completes, type
\begin{verbatim}
  su -c "python setup.py install"
\end{verbatim}
Congratulations! You have now successfully installed PyLinear.

\chapter{The \class{Array} types}
\begin{quote}
   This chapter decribes the basic array types provided by PyLinear, and
   the elementary operations available on them.
\end{quote}

The present chapter describes the module \module{pylinear.array}. It is
assumed to be imported using
\begin{verbatim}
>>> from pylinear.array import *
\end{verbatim}
in the examples.

PyLinear provides four flavors of arrays with two different element
types each.  The possible \dfn{flavors}\index{flavor} are:
\begin{itemize}
  \item dense vector\indexii{Dense}{Vector},
  \item dense matrix\indexii{Dense}{Matrix},
  \item sparse build matrix\indexiii{Sparse}{Build}{Matrix}, and
  \item sparse execute matrix\indexiii{Sparse}{Execute}{Matrix}.
\end{itemize}
From this point onwards, we will use the term \class{Array}\index{array}
to mean any type of matrix or vector. The term \class{Matrix}\index{matrix}
will refer to any of the matrix types. The term \class{Vector}\index{vector}
shall refer to only the dense vector type. Note that these do not exist as
classes as such, but we will pretend that they do.

The supported element types are
\begin{itemize}
  \item double precision (i.e. 64-bit) real, and
  \item double precision (i.e. 2x64-bit) complex.
\end{itemize}

PyLinear only allows vectors and matrices (i.e.  objects with one and
two index dimensions, but no tensors or other higher-dimensional
objects). There is only one vector flavor, but there are three
different flavors of matrices with different performance and memory
characteristics. \emph{Dense} matrices store \var{m}-by-\var{n}
elements in a two-dimensional grid of \var{m} rows and \var{n}
columns. They are used for small matrices or those which have mostly
non-zero elements. Contrast this with the sparse types, which are
typically used for matrices consisting of mostly zero
elements. \emph{Sparse build} matrices store their elements a list of
\samp{(i, j, a[i,j])}, to which new elements are simply appended,
which is very fast. This list is typically unsorted, but may have to
be sorted by \var{i} and \var{j} for multiplication, element access or
element removal, which makes these operations pretty
slow. Consequently, this flavor is typically used for the assembly of
large sparse matrices. It is then converted to the \emph{sparse
execute} flavor for fast matrix multiplication. This flavor uses a
standard compressed column format for fast linear algebra operations.

Each of the flavors is represented by a symbolic constant:

\begin{tableii}{l|l}{constant}{Constant}{Corresponding Flavor}
  \lineii{Vector}
         {Specifies the dense vector flavor.}
  \lineii{DenseMatrix}
         {Specifies the dense matrix flavor.}
  \lineii{SparseBuildMatrix}
         {Specifies the sparse build matrix flavor.}
  \lineii{SparseExecuteMatrix}
         {Specifies the sparse execute matrix flavor.}
\end{tableii}

Likewise, each of the element types has its own symbolic constant.
These constants are usually called \indexii{type}{code}\index{typecode}\dfn{typecodes}.

\begin{tableii}{l|l}{constant}{Constant}{Corresponding Element Type}
  \lineii{Float64}
         {Specifies the 64-bit real element type.}
  \lineii{Complex64}
         {Specifies the 2x64-bit complex element type.}
  \lineii{Float}
         {Specifies the machine-native C++ \ctype{double} element type.
          Currently an alias for \constant{Float64}.}
  \lineii{Complex}
         {Specifies the machine-native C++ \ctype{std::complex<double>} 
          element type. Currently an alias for \constant{Complex64}.}
\end{tableii}

Despite some dissimilarity, PyLinear's matrix layer attempts to be
mostly compatible with Numerical Python.

% --------------------------------------------------------------------------
\section{Creating \class{Array}s}

The following functions in the module \module{pylinear.array} permit
the creation of new \class{Array}s:

\begin{funcdesc}{array}{sequence, typecode=None}
   There are many ways to create arrays. The most basic one is the use of the
   \function{array} function:
\begin{verbatim}
>>> a = array([1.2, 3.5, -1])
\end{verbatim}
   to make sure this worked, do:
\begin{verbatim}
>>> print a
[3](1.2,3.5,-1)
\end{verbatim}
   The \function{array} function takes several arguments --- the first
   one is the values, which can be a Python sequence object (such as a
   list or a tuple).  The optional argument \code{type} specifies the
   element type of the matrix. If omitted, as in the example above,
   Python tries to find the best data type which can represent all the
   elements. \function{array} always creates dense matrices or
   vectors, depending on the \dfn{dimensionality}\index{dimension} of
   the input data.  (The dimension of the data is 1 for a list, 2 for a
   list of lists, and so on.  1-dimensional data will be converted to
   vectors, 2-dimensional data to matrices.)
   
   Since the elements we gave our example were two floats and one integer, it
   chose \class{Float64} as the type of the resulting array. One can specify
   unequivocally the \code{type} of the elements---this is especially 
   useful when, for example, one wants an array contains complex numbers even
   though all of its input elements are reals:
\begin{verbatim}
>>> x,y,z = 1,2,3
>>> a = array([x,y,z])                  # reals are enough for 1, 2 and 3
[3](1,2,3)
>>> print a
>>> a = array([x,y,z], type=Complex64)    # not the default type
>>> print a
[3]((1,0),(2,0),(3,0))
\end{verbatim}
    Note that in NumPy, \function{array} takes a few more arguments, such as
    \code{copy}, \code{savespace}, and \code{shape}. These are not supported.
\end{funcdesc}

\begin{funcdesc}{sparse}{mapping, shape=None, typecode=None, flavor=SparseBuildMatrix}
  This function creates a (not necessarily sparse) \class{Matrix} of
  the given \code{shape}, \code{typecode}, and \code{flavor} based on
  a sparse representation its entries. At present, it cannot create
  \class{Vector}s. The sparse representation consists of a dictionary
  of dictionaries, whose keys are the row indices for the outer dictionary,
  and the column indices for the inner one.

  If the \code{shape} parameter is unspecified, the shape is specified by
  the largest row and column indices seen in examining the \code{mapping}.
  If the \code{typecode} is unspecified, \function{sparse} uses the same
  logic as \function{data} to determine it.
\end{funcdesc}

\begin{funcdesc}{asarray}{seq, typecode, flavor = None}
  This function converts scalars, lists and tuples to an
  \class{Array} type, if possible. It passes \class{Array}s through,
  making copies only to convert types.  In any other case a
  \class{TypeError} is raised.
\end{funcdesc}

\begin{funcdesc}{zeros}{shape, typecode, flavor = None}
  \function{zeros} creates an \class{Array} of the given \var{shape},
  \var{typecode} and \var{flavor} which is filled with zeros. See the
  \member{shape} attribute in Section \ref{sec:arrayproperties} for
  information on the \var{shape} parameter.
\end{funcdesc}

\begin{funcdesc}{ones}{shape, typecode, flavor = None}
  \function{ones} creates an \class{Array} of the given \var{shape},
  \var{typecode} and \var{flavor} which is filled with ones. See the
  \member{shape} attribute in Section \ref{sec:arrayproperties} for
  information on the \var{shape} parameter.
\end{funcdesc}

\begin{funcdesc}{identity}{n, typecode, flavor = None}
  \function{identity} creates a \class{Matrix} of shape \code{(n,n)} and
  the given \var{typecode} and \var{flavor} which is filled with
  zeros and has ones on the diagonal, a type of matrix otherwise
  known as an identity matrix.
\end{funcdesc}

\class{Array}s also have efficient support for pickling. So, unpickling
a previously pickled \class{Array} is another way to create one.

% --------------------------------------------------------------------------
\section{Accessing \class{Array} properties}

\label{sec:arrayproperties}
An \class{Array} has the following meta-data attributes:

\begin{memberdesc}[Array]{shape}
  Reading the \member{shape} attribute gets the shape tuple, that is,
  a tuple of length equal to the array's dimension specifying the
  dimensions of the matrix.  For a vector, this is a singleton
  containing an integer, for a matrix, this is a pair containing the
  number of rows and columns, in this order.  Assigning a value to the
  \member{shape} attribute is not supported.
\end{memberdesc}

\begin{memberdesc}[Array]{flavor}
  Reading the \member{flavor} attribute gets the flavor of the given
  matrix. Assigning a value to the \member{flavor} attribute is not supported.
\end{memberdesc}

An \class{Array} has the following meta-data-returning methods:

\begin{methoddesc}[Array]{typecode}{}
  Returns the typecode of the matrix.
\end{methoddesc}

% --------------------------------------------------------------------------
\section{Accessing and modifying \class{Array} data}

An \class{Array} has the following data attributes:

\begin{memberdesc}[Array]{real}
  Reading this attribute obtains a copy of the real part of the matrix.
  For real matrices, the matrix is simply copied.

  Returns a view, not a copy, in NumPy. Only on complex arrays in NumPy.
\end{memberdesc}

\begin{memberdesc}[Array]{imaginary}
  Reading this attribute obtains a copy of the imaginary part of the matrix.
  For real matrices, a zero matrix of the same size is returned.

  Returns a view, not a copy, in NumPy. Only on complex arrays in NumPy.
\end{memberdesc}

In addition to the \class{Array} data attributes, \class{Matrix} types
offer the following:

\begin{memberdesc}[Matrix]{T}
  Returns a real-transpose copy of the matrix.

  Does not exist in NumPy.
\end{memberdesc}

\begin{memberdesc}[Matrix]{H}
  Returns a conjugate-transpose copy of the matrix.
  Identical to \member{T} for real matrices.

  Does not exist in NumPy.
\end{memberdesc}

Naturally, PyLinear will also support indexing for reading and writing
on \class{Array}s. Typically, matrices are indexed by 2-tuples,
whereas vectors are indexed by single values. Indexing a matrix with a
single value will return the entire row. Regular Python slice syntax
is supported, so that you can write \code{a[3:17]}, \code{a[1:3,5:9]},
or even \code{a[::-1]}. Unlike Python lists, \class{Array}s may not be
resized using slice assignments. Like in the rest of Python, yet
unlike NumPy, slices return copies, not views of the corresponding
data.

The following methods are available on PyLinear's \class{Array} types:
% XXX sum
% abs_square_sum intentionally undocumented
% XXX iter
% XXX indices
% XXX add_scattered
% XXX copy
% XXX sort
% XXX set_element
% XXX add_element
% XXX set_element_past_end
% XXX complete_index1_data

PyLinear also supports a number of data access functions that resemble
NumPy's functionality and are there mainly to ease porting:

\begin{funcdesc}{diagonal}{matrix, offset=0}
  Returns the diagonal of \var{matrix} as a vector, or the \var{offset}th
  super- (for \code{offset>0}) or sub-diagonal (for \code{offset<0}).
\end{funcdesc}
\begin{funcdesc}{take}{matrix, indices, axis=0}
  Assembles an \class{Array} from the entries of the \class{Array}
  listed in \var{indices}, which must be simple numbers. \var{axis}
  specifies the axis along which the indices are taken.
\end{funcdesc}
\begin{funcdesc}{matrixmultiply}{op1, op2}
  Equivalent to \code{op1*op2}.
\end{funcdesc}
\begin{funcdesc}{innerproduct}{op1, op2}
  Equivalent to \code{op1*op2}.
\end{funcdesc}
\begin{funcdesc}{outerproduct}{op1, op2}
  Equivalent to \code{op1 <<outerproduct>> op2}.
\end{funcdesc}
\begin{funcdesc}{transpose}{op1}
  Equivalent to \code{op1.T}.
\end{funcdesc}
\begin{funcdesc}{hermite}{op1}
  Equivalent to \code{op1.H}.
\end{funcdesc}
\begin{funcdesc}{trace}{matrix, offset=0}
  Returns the sum of the \var{offset}th diagonal. See \function{diagonal}
  for details of the meaning of \var{offset}.
\end{funcdesc}
% --------------------------------------------------------------------------
\section{Elementary computations with matrices}
\class{Array}s support the operators \code{+} (binary), \code{+=},
\code{-} (binary), \code{-=}, as well as \code{+} (unary) and \code{-}
(unary) with their elementwise meanings as you would expect them.
Multiplication and division are also supplied, but have slightly more
intricate meanings, as discussed in Section
\ref{subsec:arraymultiplication}.

\subsection{Type promotion}
\label{subsec:arraypromotion}

If binary operators or ufuncs (see Section \ref{subsec:ufuncs})
are applied to arrays of non-matching flavor or typecode,
the operands are promoted to a common type. (For the case of
non-matching dimension, see Section \ref{subsec:arraybroadcast}
for broadcasting rules.)

If the only mismatch is in typecode, one argument array
is cast upward in the type hierarchy (e.g. from integer to real,
from real to complex) in order to match the other.

If there is also a mismatch in flavor, the matrix with the typecode
which is higher up in the type hierarchy determines the flavor of the
result.

\subsection{Broadcasting}
\label{subsec:arraybroadcast}

The binary elementwise operators as well as all the binary ufuncs (see
Section \ref{subsec:ufuncs}) accept argument pairs where one argument
has lesser dimension than the other. In this case, the missing dimensions
are \index{broadcasting}\dfn{broadcast} across the remainder of the
matrix. If the lesser-dimension argument is a scalar, this is easy to
explain: It is treated like an array of the right size filled with
that scalar. If it is a vector, it is treated like a matrix filled
with rows consisting of the given vector.

All of this can only work if the corresponding \class{Array} sizes 
match.

\subsection{Universal Functions}
\label{subsec:ufuncs}

PyLinear sports a few so-called \dfn{Universal Functions},
\indexii{Universal}{Function} some of which are \emph{unary},
while others are \emph{binary}. (The lengthy term universal function is
often abbreviated to \dfn{ufunc}\index{ufunc}.) Universal functions
generally apply a certain functionality to each element in an
array. For example, the \function{sin} universal function computes the
sine of each of the given array's entries, and returns the processed
matrix, which will be of the same size, flavor, and typecode. Binary
universal functions receive two \class{Array}s of equal size as
arguments, apply a binary function (such as, for example, addition or
multiplication) to each pair of entries of the two \class{Array}s,
pairing the entries at the same location in each \class{Array}, and
return an \class{Array} with the results. Binary universal functions
obey type promotion laws as laid out in section
\ref{subsec:arraypromotion}.

The following unary universal functions exist:

\begin{funcdesc}{conjugate}{array}
  Returns the complex-conjugate of the given \class{Array}. Simply
  copies real matrices.
\end{funcdesc}
\begin{funcdesc}{cos}{array}
  Returns the element-wise cosine of the given \class{Array}.
\end{funcdesc}
\begin{funcdesc}{cosh}{array}
  Returns the element-wise hyperbolic cosine of the given
  \class{Array}.
\end{funcdesc}
\begin{funcdesc}{exp}{array}
  Returns the element-wise natural exponential of the given
  \class{Array}. 

  \emph{WARNING:} This is not matrix exponentiation.
\end{funcdesc}
\begin{funcdesc}{log}{array}
  Returns the element-wise natural logarithm of the given
  \class{Array}.
\end{funcdesc}
\begin{funcdesc}{log10}{array}
  Returns the element-wise base-10 logarithm of the given
  \class{Array}.
\end{funcdesc}
\begin{funcdesc}{sin}{array}
  Returns the element-wise sine of the given \class{Array}.
\end{funcdesc}
\begin{funcdesc}{sinh}{array}
  Returns the element-wise hyperbolic sine of the given \class{Array}.
\end{funcdesc}
\begin{funcdesc}{sqrt}{array}
  Returns the element-wise square root of the given \class{Array}.
\end{funcdesc}
\begin{funcdesc}{tan}{array}
  Returns the element-wise tangent of the given \class{Array}.
\end{funcdesc}
\begin{funcdesc}{tanh}{array}
  Returns the element-wise hyperbolic tangent of the given \class{Array}.
\end{funcdesc}
\begin{funcdesc}{floor}{array}
  Returns the element-wise floor of the given \class{Array}.
\end{funcdesc}
\begin{funcdesc}{ceil}{array}
  Returns the element-wise ceiling of the given \class{Array}.
\end{funcdesc}
\begin{funcdesc}{argument}{array}
  Returns the element-wise complex argument of the given \class{Array}.
  Resulting matrix consists of values of zero and $\pi$ for real matrices.
\end{funcdesc}
\begin{funcdesc}{absolute}{array}
  Returns the element-wise absolute value of the given \class{Array}.
\end{funcdesc}

The following binary universal functions exist:

\begin{funcdesc}{add}{op1, op2}
  Returns the element-wise sum of the given \class{Array}s. Obeys
  broadcasting (see Section \ref{subsec:arraybroadcasting}) and type
  promotion (see Section \ref{subsec:arraypromotion}) laws.

  Equivalent to the \code{+} operator.
\end{funcdesc}
\begin{funcdesc}{subtract}{op1, op2}
  Returns the element-wise difference of the given \class{Array}s. Obeys
  broadcasting (see Section \ref{subsec:arraybroadcasting}) and type
  promotion (see Section \ref{subsec:arraypromotion}) laws.

  Equivalent to the \code{-} operator.
\end{funcdesc}
\begin{funcdesc}{multiply}{op1, op2}
  Returns the element-wise product of the given \class{Array}s. Obeys
  broadcasting (see Section \ref{subsec:arraybroadcasting}) and type
  promotion (see Section \ref{subsec:arraypromotion}) laws.

  \emph{NOT} equivalent to the \code{*} operator, except in the scalar
  case.
\end{funcdesc}
\begin{funcdesc}{divide}{op1, op2}
  Returns the element-wise quotient of the given \class{Array}s. Obeys
  broadcasting (see Section \ref{subsec:arraybroadcasting}) and type
  promotion (see Section \ref{subsec:arraypromotion}) laws.

  \emph{NOT} equivalent to the \code{/} operator, except in the scalar
  case.
\end{funcdesc}
\begin{funcdesc}{power}{op1, op2}
  Returns the element-wise power \code{op1[i]**op2[i]} of the given
  \class{Array}s. Obeys broadcasting (see Section
  \ref{subsec:arraybroadcasting}) and type promotion (see Section
  \ref{subsec:arraypromotion}) laws.

  \emph{NOT} equivalent to the \code{**} operator, except in the scalar
  case.
\end{funcdesc}
\begin{funcdesc}{maximum}{op1, op2}
  Returns the element-wise maximum of the given
  \class{Array}s. Obeys broadcasting (see Section
  \ref{subsec:arraybroadcasting}) and type promotion (see Section
  \ref{subsec:arraypromotion}) laws.

  For complex matrices, the maximum is found based on the real part.
\end{funcdesc}
\begin{funcdesc}{minimum}{op1, op2}
  Returns the element-wise minimum of the given
  \class{Array}s. Obeys broadcasting (see Section
  \ref{subsec:arraybroadcasting}) and type promotion (see Section
  \ref{subsec:arraypromotion}) laws.

  For complex matrices, the minimum is found based on the real part.
\end{funcdesc}

Additional universal function methods, such as \member{reduce}, as they are
found in NumPy, are not (yet) supported in PyLinear.

\subsection{Multiplication semantics}
\label{subsec:arraymultiplication}

This section explains what the value of \code{a*b} is, where at least
one of \code{a} and \code{b} is an \class{Array}.

If the other operand is a scalar (it doesn't matter which), the
result will be the element-wise product of the array with that scalar.

If both operands are \class{Vector}s, \code{a*b} computes the inner
product of both vectors. Note that in the complex case no complex
conjugates are taken. If you require them, use the expression
\code{a*b.H}.

If \code{a} is a \class{Vector} and \code{b} is a \class{Matrix},
\code{a*b} will result in $b^Ta$, using the conventional matrix-vector
product.

If \code{a} is a \class{Matrix} and \code{b} is a \class{Vector},
\code{a*b} will result in $a b$, using the conventional matrix-vector
product.

If both \code{a} and \code{b} are \class{Matrix} types,
\code{a*b} will result in $a b$, using the conventional matrix-matrix
product.

All these explanations also apply to the inplace multiplication
operator \code{*=}.

All multiplication operators obey type promotion rules as laid out
in Section \ref{subsec:arraypromotion}.

\subsection{Other \class{Array} operators}

The following expressions are also valid:
\begin{itemize} 
\item \code{matrix**n}

  Computes the \var{n}th power of \var{matrix}. \var{n} must be
  integer, but may be negative. Only for dense matrices.

\item \code{scalar/matrix}

  Computes the \var{scalar} multiple of the inverse of
  \var{matrix}. Only for dense matrices.

  Do not use code like \code{1/a*x} to solve linear systems of
  equations; besides being slow, this tends to yield imprecise
  results. Instead, use the \code{<<solve>>} pseudo-operator.

  Use of this operator will fail unless the module
  \module{pylinear.operation} is available.

\item \code{matrix <<solve>> vector}

  Returns the solution of the linear system of equations \code{matrix*x=vector}.
  Available for dense and sparse execute matrix types of \var{matrix}.

  Use of this operator will fail unless the module
  \module{pylinear.operation} is available.

\item \code{vector1 <<outer>> vector2}
  
  Computes the outer product of \code{vector1} and \code{vector2}.
\end{itemize}

\chapter{Matrix-Free Methods}
\label{cha:matrixfree}

\begin{quote}
   This chapter introduces the notion of an \class{Operator}, which
   is PyLinear's way of expressing matrix-free methods.
\end{quote}

\section{The \class{Operator} concept}

\begin{classdesc*}{Operator}
  An \class{Operator}\index{Operator} is a (typically linear) mapping
  of one vector to another. A matrix is a particularly prominent
  example of this, but \class{Operator}s are mainly used to represent
  vector-to-vector mappings for which no matrix is available (or too
  expensive to compute explicitly).
\end{classdesc*}

Given this single purpose, an \class{Operator} has a pretty 
simple interface:
\begin{memberdesc}[Operator]{shape}
  Returns a tuple \code{(h,w)}, which, in analogy to a \class{Matrix},
  specifies the sizes of the vectors received and returned by the
  \class{Operator}.
\end{memberdesc}
\begin{methoddesc}[Operator]{typecode}{}
  Returns the typecode (see \ref{sec:types-and-flavors}) of the
  \class{Vector}s that this \class{Operator} operates on. This is also
  the typecode of the \class{Vector}qs returned by the operations of this
  \class{Operator}. For technical reasons, the two need to match.
\end{methoddesc}
\begin{methoddesc}[Operator]{apply}{before, after}
  This method operates on the \class{Vector} \code{before} and returns
  the result of the operation in \code{after}. \code{after}
  needs to be a properly-sized \class{Vector}. Its initial values
  typically do not matter, but may be used as starting guesses,
  for example by iterative solvers. Initializing after to
  all zeroes is always acceptable.
\end{methoddesc}

\section{\class{Operator}s Form an Algebra}

On top of this simple interface of an \class{Operator}, PyLinear 
provides a layer of convenience functions that facilitate the
creation of derived instances, among other things.

\opindex{()}For an \class{Operator} \code{A}, saying \code{A(x)}
with a properly sized and typed \class{Vector} \code{x} will
return the result of applying \code{A} to \vector{x}, via the


\opindex{+}For two \class{Operator}s \code{A} and \code{B}, you may write
\code{A+B} to obtain an \class{Operator} mapping that will
perform the operation \code{A(x)+B(x)}. The operator \code{-} works
in an analogous fashion.

\opindex{*}For two \class{Operator}s \code{A} and \code{B}, you may
write \code{A*B} to obtain an \class{Operator} mapping that will
perform the composed operation \code{A(B(x))}. You may also say
\code{a*B} or {B*a} with an \class{Operator} \code{B} and a scalar
\code{a}, and will obtain an \class{Operator} that performs
\code{a*B(x)}. A unary minus \code{-A} returns the negated
operator.

\section{Types of \class{Operator}s}

\section{Implementing your own \class{Operator}s}

FIXME \dots

For information on implementing your own operators in C++, please
refer to Chapter \ref{cha:extending}.

\chapter{Matrix-Free Methods}
\label{cha:numerics}

\begin{quote}
   This chapter introduces the numerical algorithms available
   in PyLinear.
\end{quote}

PyLinear features six different modules of numerical algorithms:
\begin{itemize}
  \item \module{pylinear.operation} uses the previously introduced 
    notion of an \class{Operator} and offers several implementations 
    of the concept. It is also the main module of linear
    algebra computational routines in PyLinear. It offers a
    comprehensive set of linear algebra primitives, such as
    determinants, decompositions, linear solves, eigenvalue finding
    and the like.
  \item \module{pylinear.linear_algebra} is a compatibility module 
    which aims for complete exchangeability with NumPy's \module{LinearAlgebra}.
    It offers a high-level subset of \module{pylinear.operation} 
    with less exposed detail.
  \item \module{pylinear.mpi} will provide an interface between
   MPI and PyLinear, but is not yet written.
  \item \module{pylinear.toybox} serves as a staging area for
    the above modules and has an unspecified interface that may 
    change at any time. Look in the source to find experimental
    algorithms that may solve your problems, but be warned that
    these may disappear or change at any time.
\end{itemize}

\section{Matrix computations}
\section{Convenient helpers}
\section{Parallel PyLinear}

Does not exist yet. :)

\chapter{Extending PyLinear}
\label{cha:extending}

\begin{quote}
   This chapter ...
\end{quote}

FIXME
 

\chapter{Differences to NumPy and numarray}
\label{cha:diff-pylinear-numpy}

\begin{quote}
   This chapter outlines the differences between PyLinear and the packages
   Numerical Python and numarray.
\end{quote}

Unlike NumPy, PyLinear does \emph{not} allow more than two or less
than one index dimension, i.e. it only suports objects with one and
two indices, also known as vectors and matrices. In fact, even vectors
and matrices are different types internally, while NumPy glosses over
these differences and makes them all a single \emph{array type}. This
makes sense since NumPy's focus is on array-shaped data, such as
images and measurements, while PyLinear's focus is on linear algebra.

\begin{itemize}
  \item Ufunc methods are not supported.
  \item Slices copy, do not return views.
\end{itemize}


\appendix

\chapter{Acknowledgements}
\label{cha:ack}

PyLinear owes much to the heroes who fleshed out Numerical Python, numarray,
numpy and their corresponding interfaces. In particular, some parts of this
manual are shamelessly borrowed from numarray, as are a few docstrings.

PyLinear was written as part of a Diplom thesis at the Institut f�r Angewandte
Mathematik at Universit�t Karlsruhe (TH), Germany. I am grateful that my
advisor, Prof. Dr. Willy D�rfler, gave me the freedom to choose to write my own
matrix package as part of my thesis. The package was significantly enhanced and
released to the public during a paid research stay at his institute, whose
support I gratefully acknowledge.

PyLinear has also benefitted from discussions I had with Roman Geus of the Paul
Scherrer Institute in early stages of the project.

Last, but not least, PyLinear would not even exist if it weren't for Python and
the Boost C++ libraries.


% Complete documentation on the extended LaTeX markup used for Python
% documentation is available in ``Documenting Python'', which is part
% of the standard documentation for Python.  It may be found online
% at:
%
%     http://www.python.org/doc/current/doc/doc.html

\documentclass[hyperref]{manual}

% latex2html doesn't know [T1]{fontenc}, so we cannot use that:(

\usepackage{amsmath}
\usepackage[latin1]{inputenc}
\usepackage{textcomp}

% provide a cross-linking command for the index
%begin{latexonly}
\newcommand*\see[2]{\protect\seename #1}
\newcommand*{\seename}{$\to$}
%end{latexonly}


% mark internal comments
% for any published version switch to the second (empty) definition of the macro!
% \newcommand{\remark}[1]{(\textbf{Note to authors: #1})}
\newcommand{\remark}[1]{}


\title{PyLinear\\An Open Source project}

\author{Andreas Kl\"ockner}

\authoraddress{Institut f\"ur Angewandte Mathematik II, Englerstra\ss e 2\\
   76xxx Karlsruhe}

% I use date to indicate the manual-updates,
% release below gives the matching software version.
\date{March 10, 2004}
\release{0.1}                 % (software) release version;
\setshortversion{0.1}         % this is used to define the \version macro

\makeindex                      % tell \index to actually write the .idx file



\begin{document}

\maketitle

% This makes the contents more accessible from the front page of the HTML.
\ifhtml
\part*{General}
\chapter*{Front Matter}
\label{front}
\fi


\tableofcontents


\chapter{Hey du!}
and the functions that operate upon them.

\label{part:numerical-python}

\declaremodule{extension}{numarray}
\moduleauthor{The numarray team}{numpy@lists.sourceforge.net}
\modulesynopsis{Numerics}

\appendix
%begin{latexonly}
%end{latexonly}

%% Complete documentation on the extended LaTeX markup used for Python
% documentation is available in ``Documenting Python'', which is part
% of the standard documentation for Python.  It may be found online
% at:
%
%     http://www.python.org/doc/current/doc/doc.html

\documentclass[hyperref]{manual}

% latex2html doesn't know [T1]{fontenc}, so we cannot use that:(

\usepackage{amsmath}
\usepackage[latin1]{inputenc}
\usepackage{textcomp}

% provide a cross-linking command for the index
%begin{latexonly}
\newcommand*\see[2]{\protect\seename #1}
\newcommand*{\seename}{$\to$}
%end{latexonly}


% mark internal comments
% for any published version switch to the second (empty) definition of the macro!
% \newcommand{\remark}[1]{(\textbf{Note to authors: #1})}
\newcommand{\remark}[1]{}


\title{PyLinear\\An Open Source project}

\author{Andreas Kl\"ockner}

\authoraddress{Institut f\"ur Angewandte Mathematik II, Englerstra\ss e 2\\
   76xxx Karlsruhe}

% I use date to indicate the manual-updates,
% release below gives the matching software version.
\date{March 10, 2004}
\release{0.1}                 % (software) release version;
\setshortversion{0.1}         % this is used to define the \version macro

\makeindex                      % tell \index to actually write the .idx file



\begin{document}

\maketitle

% This makes the contents more accessible from the front page of the HTML.
\ifhtml
\part*{General}
\chapter*{Front Matter}
\label{front}
\fi


\tableofcontents


\chapter{Hey du!}
and the functions that operate upon them.

\label{part:numerical-python}

\declaremodule{extension}{numarray}
\moduleauthor{The numarray team}{numpy@lists.sourceforge.net}
\modulesynopsis{Numerics}

\appendix
%begin{latexonly}
%end{latexonly}

%% Complete documentation on the extended LaTeX markup used for Python
% documentation is available in ``Documenting Python'', which is part
% of the standard documentation for Python.  It may be found online
% at:
%
%     http://www.python.org/doc/current/doc/doc.html

\documentclass[hyperref]{manual}

% latex2html doesn't know [T1]{fontenc}, so we cannot use that:(

\usepackage{amsmath}
\usepackage[latin1]{inputenc}
\usepackage{textcomp}

% provide a cross-linking command for the index
%begin{latexonly}
\newcommand*\see[2]{\protect\seename #1}
\newcommand*{\seename}{$\to$}
%end{latexonly}


% mark internal comments
% for any published version switch to the second (empty) definition of the macro!
% \newcommand{\remark}[1]{(\textbf{Note to authors: #1})}
\newcommand{\remark}[1]{}


\title{PyLinear\\An Open Source project}

\author{Andreas Kl\"ockner}

\authoraddress{Institut f\"ur Angewandte Mathematik II, Englerstra\ss e 2\\
   76xxx Karlsruhe}

% I use date to indicate the manual-updates,
% release below gives the matching software version.
\date{March 10, 2004}
\release{0.1}                 % (software) release version;
\setshortversion{0.1}         % this is used to define the \version macro

\makeindex                      % tell \index to actually write the .idx file



\begin{document}

\maketitle

% This makes the contents more accessible from the front page of the HTML.
\ifhtml
\part*{General}
\chapter*{Front Matter}
\label{front}
\fi


\tableofcontents


\chapter{Hey du!}
and the functions that operate upon them.

\label{part:numerical-python}

\declaremodule{extension}{numarray}
\moduleauthor{The numarray team}{numpy@lists.sourceforge.net}
\modulesynopsis{Numerics}

\appendix
%begin{latexonly}
%end{latexonly}

%\input{pylinear.ind}

\end{document}


\end{document}


\end{document}


\end{document}
