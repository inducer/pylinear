% Complete documentation on the extended LaTeX markup used for Python
% documentation is available in ``Documenting Python'', which is part
% of the standard documentation for Python.  It may be found online
% at:
%
%     http://www.python.org/doc/current/doc/doc.html

\documentclass[hyperref]{manual}

% latex2html doesn't know [T1]{fontenc}, so we cannot use that:(

\usepackage{amsmath}
\usepackage[latin1]{inputenc}
\usepackage{textcomp}

% provide a cross-linking command for the index
%begin{latexonly}
\newcommand*\see[2]{\protect\seename #1}
\newcommand*{\seename}{$\to$}
%end{latexonly}


% mark internal comments
% for any published version switch to the second (empty) definition of the macro!
% \newcommand{\remark}[1]{(\textbf{Note to authors: #1})}
\newcommand{\remark}[1]{}


\title{PyLinear\\An Open Source project}

\author{Andreas Kl\"ockner}

\authoraddress{Institut f\"ur Angewandte Mathematik II, Englerstra\ss e 2\\
   76xxx Karlsruhe}

% I use date to indicate the manual-updates,
% release below gives the matching software version.
\date{March 10, 2004}
\release{0.1}                 % (software) release version;
\setshortversion{0.1}         % this is used to define the \version macro

\makeindex                      % tell \index to actually write the .idx file



\begin{document}

\maketitle

% This makes the contents more accessible from the front page of the HTML.
\ifhtml
\part*{General}
\chapter*{Front Matter}
\label{front}
\fi


\tableofcontents


\chapter{Hey du!}
and the functions that operate upon them.

\label{part:numerical-python}

\declaremodule{extension}{numarray}
\moduleauthor{The numarray team}{numpy@lists.sourceforge.net}
\modulesynopsis{Numerics}

\appendix
%begin{latexonly}
%end{latexonly}

%% Complete documentation on the extended LaTeX markup used for Python
% documentation is available in ``Documenting Python'', which is part
% of the standard documentation for Python.  It may be found online
% at:
%
%     http://www.python.org/doc/current/doc/doc.html

\documentclass[hyperref]{manual}

% latex2html doesn't know [T1]{fontenc}, so we cannot use that:(

\usepackage{amsmath}
\usepackage[latin1]{inputenc}
\usepackage{textcomp}

% provide a cross-linking command for the index
%begin{latexonly}
\newcommand*\see[2]{\protect\seename #1}
\newcommand*{\seename}{$\to$}
%end{latexonly}


% mark internal comments
% for any published version switch to the second (empty) definition of the macro!
% \newcommand{\remark}[1]{(\textbf{Note to authors: #1})}
\newcommand{\remark}[1]{}


\title{PyLinear\\An Open Source project}

\author{Andreas Kl\"ockner}

\authoraddress{Institut f\"ur Angewandte Mathematik II, Englerstra\ss e 2\\
   76xxx Karlsruhe}

% I use date to indicate the manual-updates,
% release below gives the matching software version.
\date{March 10, 2004}
\release{0.1}                 % (software) release version;
\setshortversion{0.1}         % this is used to define the \version macro

\makeindex                      % tell \index to actually write the .idx file



\begin{document}

\maketitle

% This makes the contents more accessible from the front page of the HTML.
\ifhtml
\part*{General}
\chapter*{Front Matter}
\label{front}
\fi


\tableofcontents


\chapter{Hey du!}
and the functions that operate upon them.

\label{part:numerical-python}

\declaremodule{extension}{numarray}
\moduleauthor{The numarray team}{numpy@lists.sourceforge.net}
\modulesynopsis{Numerics}

\appendix
%begin{latexonly}
%end{latexonly}

%% Complete documentation on the extended LaTeX markup used for Python
% documentation is available in ``Documenting Python'', which is part
% of the standard documentation for Python.  It may be found online
% at:
%
%     http://www.python.org/doc/current/doc/doc.html

\documentclass[hyperref]{manual}

% latex2html doesn't know [T1]{fontenc}, so we cannot use that:(

\usepackage{amsmath}
\usepackage[latin1]{inputenc}
\usepackage{textcomp}

% provide a cross-linking command for the index
%begin{latexonly}
\newcommand*\see[2]{\protect\seename #1}
\newcommand*{\seename}{$\to$}
%end{latexonly}


% mark internal comments
% for any published version switch to the second (empty) definition of the macro!
% \newcommand{\remark}[1]{(\textbf{Note to authors: #1})}
\newcommand{\remark}[1]{}


\title{PyLinear\\An Open Source project}

\author{Andreas Kl\"ockner}

\authoraddress{Institut f\"ur Angewandte Mathematik II, Englerstra\ss e 2\\
   76xxx Karlsruhe}

% I use date to indicate the manual-updates,
% release below gives the matching software version.
\date{March 10, 2004}
\release{0.1}                 % (software) release version;
\setshortversion{0.1}         % this is used to define the \version macro

\makeindex                      % tell \index to actually write the .idx file



\begin{document}

\maketitle

% This makes the contents more accessible from the front page of the HTML.
\ifhtml
\part*{General}
\chapter*{Front Matter}
\label{front}
\fi


\tableofcontents


\chapter{Hey du!}
and the functions that operate upon them.

\label{part:numerical-python}

\declaremodule{extension}{numarray}
\moduleauthor{The numarray team}{numpy@lists.sourceforge.net}
\modulesynopsis{Numerics}

\appendix
%begin{latexonly}
%end{latexonly}

%% Complete documentation on the extended LaTeX markup used for Python
% documentation is available in ``Documenting Python'', which is part
% of the standard documentation for Python.  It may be found online
% at:
%
%     http://www.python.org/doc/current/doc/doc.html

\documentclass[hyperref]{manual}

% latex2html doesn't know [T1]{fontenc}, so we cannot use that:(

\usepackage{amsmath}
\usepackage[latin1]{inputenc}
\usepackage{textcomp}

% provide a cross-linking command for the index
%begin{latexonly}
\newcommand*\see[2]{\protect\seename #1}
\newcommand*{\seename}{$\to$}
%end{latexonly}


% mark internal comments
% for any published version switch to the second (empty) definition of the macro!
% \newcommand{\remark}[1]{(\textbf{Note to authors: #1})}
\newcommand{\remark}[1]{}


\title{PyLinear\\An Open Source project}

\author{Andreas Kl\"ockner}

\authoraddress{Institut f\"ur Angewandte Mathematik II, Englerstra\ss e 2\\
   76xxx Karlsruhe}

% I use date to indicate the manual-updates,
% release below gives the matching software version.
\date{March 10, 2004}
\release{0.1}                 % (software) release version;
\setshortversion{0.1}         % this is used to define the \version macro

\makeindex                      % tell \index to actually write the .idx file



\begin{document}

\maketitle

% This makes the contents more accessible from the front page of the HTML.
\ifhtml
\part*{General}
\chapter*{Front Matter}
\label{front}
\fi


\tableofcontents


\chapter{Hey du!}
and the functions that operate upon them.

\label{part:numerical-python}

\declaremodule{extension}{numarray}
\moduleauthor{The numarray team}{numpy@lists.sourceforge.net}
\modulesynopsis{Numerics}

\appendix
%begin{latexonly}
%end{latexonly}

%\input{pylinear.ind}

\end{document}


\end{document}


\end{document}


\end{document}
